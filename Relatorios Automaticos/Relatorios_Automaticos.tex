% !TEX TS-program = pdflatex
% !TEX encoding = UTF-8 Unicode

% This is a simple template for a LaTeX document using the "article" class.
% See "book", "report", "letter" for other types of document.

\documentclass[12pt]{article} % use larger type; default would be 10pt

% Para a separação de sílabas em Português
\usepackage[utf8]{inputenc} % set input encoding (not needed with XeLaTeX)
\usepackage[brazil]{babel}
\usepackage{ae} % para o pdf ficar com fontes melhores
%\usepackage[numbers]{natlib} não funciona


\usepackage[T1]{fontenc}

%\usepackage{endnotes}
%\let\footnote=\endnote

%%% Examples of Article customizations
% These packages are optional, depending whether you want the features they provide.
% See the LaTeX Companion or other references for full information.

%%% PAGE DIMENSIONS
\usepackage{geometry} % to change the page dimensions
\geometry{a4paper} % or letterpaper (US) or a5paper or....
\geometry{margin=2.5cm} % for example, change the margins to 2 inches all round
% \geometry{landscape} % set up the page for landscape
%   read geometry.pdf for detailed page layout information

\usepackage{graphicx} % support the \includegraphics command and options

% \usepackage[parfill]{parskip} % Activate to begin paragraphs with an empty line rather than an indent

%%% PACKAGES
\usepackage{booktabs} % for much better looking tables
\usepackage{array} % for better arrays (eg matrices) in maths
\usepackage{paralist} % very flexible & customisable lists (eg. enumerate/itemize, etc.)
\usepackage{verbatim} % adds environment for commenting out blocks of text & for better verbatim
\usepackage{subfig} % make it possible to include more than one captioned figure/table in a single float
% These packages are all incorporated in the memoir class to one degree or another...

%%% HEADERS & FOOTERS
\usepackage{fancyhdr} % This should be set AFTER setting up the page geometry
\pagestyle{fancy} % options: empty , plain , fancy
\renewcommand{\headrulewidth}{0pt} % customise the layout...
\lhead{}\chead{}\rhead{}
\lfoot{}\cfoot{\thepage}\rfoot{}

%%% SECTION TITLE APPEARANCE
\usepackage{sectsty}
% será que rmfamily = Times New Roman?? (Roman family)
\allsectionsfont{\rmfamily\mdseries\upshape} % (See the fntguide.pdf for font help)
% (This matches ConTeXt defaults)

%%% ToC (table of contents) APPEARANCE
\usepackage[nottoc,notlof,notlot]{tocbibind} % Put the bibliography in the ToC
\usepackage[titles,subfigure]{tocloft} % Alter the style of the Table of Contents
\renewcommand{\cftsecfont}{\rmfamily\mdseries\upshape}
\renewcommand{\cftsecpagefont}{\rmfamily\mdseries\upshape} % No bold!

\renewcommand{\abstract}{}
%\renewcommand{\thebibliography}{\fontsize{20}{20}} não funciona
%%% END Article customizations

%%%%%%%%%%%%%%%%%%%%%%%%%%%%%%%%%%%%%%%%%%%%%%%%%%%%%%%%
%%% The "real" document content comes below...

\title{\fontsize{15}{15}\textbf{Equipes remotas: ferramentas e práticas de gestão em um caso real}}
\author{Paulo Santos Abreu}
\date{17 de agosto de 2011} % Activate to display a given date or no date (if empty),
         % otherwise the current date is printed 



\begin{document}
\maketitle

\begin{abstract}
\fontsize{12}{12}\textit{A gestão de equipes remotas apresenta desafios ao gestor que não existiam em grupos não dispersos 
geograficamente. Ferramentas de tecnologia ajudam o gestor a lidar com os desafios de isolamento do grupo
e diferença de fusos horários, enquanto ferramentas de gestão de desempenho ajudam com a entrega de 
resultados. A experiência na gestão de uma equipe remota identificou quais ferramentas são as mais
eficazes e de que maneira elas podem ser utilizadas. Nem todas as ferramentas tecnológicas analisadas
mostraram-se práticas e o uso do conjunto apropriado depende de fatores da cultura da empresa.
}
 
Equipe; Virtual; Ferramenta

\end{abstract}

\section{\fontsize{12}{12}\textbf{Introdução}}

Em um mundo cada vez mais globalizado é comum que as empresas empreguem pessoas
que se encontram dispersas geograficamente entre cidades, estados e até mesmo países.
A gestão desse tipo de grupo apresenta desafios específicos que não estavam 
presentes na gestão de pessoas que trabalhavam todas no mesmo escritório.

Este artigo apresenta a experiência adquirida na gestão de um grupo de 6 pessoas distribuído
por 5 países na América Latina em 3 fusos horários diferentes com 3 línguas  
distintas sendo uma delas (Inglês) utilizada como a língua comum entre o grupo.
As ferramentas tecnológicas e de gestão de pessoas mais utilizadas 
são apresentadas e a descrição de como aplicá-las ao grupo é discutida.

No contexto deste artigo, equipe remota é um grupo de pessoas que trabalha em uma organização funcional (e.g. um Departamento da empresa) e que estão geograficamente dispersos. O termo equipe virtual 
 também pode ser utilizado para descrever esse grupo,
mas este artigo restringe a definição a funcionários sob a mesma organização hierárquica (organização funcional
	e equipe virtual estão definidos em \cite{pmbok}).


\section{\fontsize{12}{12}\textbf{Desafios de uma Equipe Remota}}

A gestão de uma equipe remota exige um esforço adicional
para manter a comunicação e a interação com os membros da equipe sempre fluindo. 
Os principais desafios presentes na gestão de uma equipe remota são (\cite{brake} e \cite{sen}):

\begin{description}
\item{Isolamento:} membros de uma equipe remota não podem
	simplesmente ``aparecer na sala do gestor'' para tirar uma dúvida ou ter uma discussão rápida, nem mesmo
	encontram o gestor ou os colegas no cafezinho para bater um papo. A distância física afasta as pessoas não só no sentido
	espacial mas também no sentido de relacionamento pessoal. O contato entre as
                                                pessoas fica mais restrito a temas do trabalho em horários específicos. Este distanciamento (físico e
							emocional) reduz o senso de identidade da equipe (faz com que a sensação de pertencer a um mesmo
							grupo diminua) e dificulta a construção de confiança. 

\item{Diferenças de fuso horário:} quanto maior a diferença de fuso horário entre os membros da equipe
                                                mais difícil ficam alguns itens logísticos, tais como marcar horário de reuniões. 
                                                O horário conveniente para um pode ser muito cedo ou muito tarde para
                                                outro. Além disso, o gestor pode precisar conversar com um membro da 
                                                equipe num horário no qual ele já não esteja mais disponível, e vice versa.


\item{Acompanhamento do trabalho:} a distância entre os membros da equipe e principalmente a distância do gestor dá a cada
                                                   pessoa uma grande liberdade no trabalho, permitindo uma maior flexibilidade  
		na escolha de horários e atividades realizadas. Esta liberdade tráz automaticamente
                                            a responsabilidade pela entrega dos resultados nos prazos acordados
					independentemente de se o chefe estava acompanhando ou não o andamento do trabalho.
\end{description}


Para lidar com os desafios da gestão de equipes remotas o gestor pode fazer uso de Ferramentas Tecnológicas
e Ferramentas de Gestão de Desempenho. Em seguida descrevemos cada uma dessas categorias.

\section{\fontsize{12}{12}\textbf{Ferramentas Tecnológicas}}

Existe uma diversidade muito grande de ferramentas de tecnologia no mercado e há sempre um novo
produto sendo lançado. Nenhuma ferramenta tecnológica vai resolver por si só as dificuldades
da gestão de equipes remotas. No entanto, há ferramentas que podem ajudar em muito o gestor
em seu objetivo de diminuir a dificuldade de comunicação e de distanciamento do grupo.

As ferramentas mais comuns encontradas atualmente são as seguintes.

\begin{description}

\item{E-mail e lista de distribuição (\textit{mailing list}):} parte integrante da comunicação no mundo atual o e-mail é ferramenta indispensável e incontestável
    na comunicação entre pessoas de quaisquer grupos corporativos. Para uma equipe remota a comunicação por e-mail
    é fundamental. Na família de ferramentas de e-mail, a lista de distribuição (ou \textit{mailing list}) é um tipo de
                                          conta de e-mail onde um sistema distribui automaticamente o e-mail recebido para todos
                                          aqueles cadastrados na lista. 

\item{Calendário compartilhado}: muitas vezes integrado à ferramenta de correio eletrônico, o calendário compartilhado
	permite agendar um compromisso e convidar outras pessoas automaticamente bastando incluir o endereço de
	e-mail das mesmas na lista de participantes do evento. A ferramenta envia um e-mail aos convidados que têm a opção
	de aceitar ou declinar do convite. Se o convite for aceito, aparecerá no calendário da pessoa convidada 
	o compromisso sincronizado com o compromisso no calendário do organizador. 


\item{Áudio conferência:} o uso do telefone é tão onipresente quanto o uso de e-mail e faz parte, sem dúvida, das 
    ferramentas usadas na gestão de equipes remotas. No entanto, quando um 
    telefonema precisa incluir mais de duas pessoas simultaneamente, a dificuldade em se estabelecer esta comunicação
    aumenta exponencialmente com a quantidade de participantes. A áudio
    conferência é um sistema de telefonia que permite a várias pessoas participarem de um telefonema ao mesmo tempo. 


\item{\textit{Chat} (ou mensagem instantânea / \textit{instant messaging}):} o \textit{instant messaging} (IM) é muito popular na Internet e 
    é cada vez mais utilizado também no mundo corporativo. Para uma equipe remota, o IM vem em adição à 
    comunicação por e-mail e ao uso do telefone. O uso mais comum do IM é para uma comunicação rápida onde não se procura fazer um 
    acompanhamento posterior sobre o assunto (nesse outro caso, o e-mail passa a ser a ferramenta de preferência).  
    

\item{Apresentações via Internet:} esta ferramenta está disponível através de vários fornecedores e produtos no mercado. O uso mais
    comum é compartilhar uma apresentação de \textit{slides} via Internet para que os demais participantes acompanhem
    a apresentação de forma mais fácil (é como se todos estivessem numa mesma sala assistindo a uma apresentação
    feita com um projetor). 

\item{Acesso remoto a sistemas:} acesso remoto aos sistemas da empresa 
    (tais como: intranet, servidores de arquivo, bases de dados, e-mail, chat, dentre outros) ocorre quando
    a pessoa tem a capacidade de acessar um ou mais dos sistemas utilizando-se de uma VPN (rede de comunicação 
	privada que se cria utilizando-se criptografia sobre uma rede de acesso público - tal como a Internet - 
		para se acessar uma outra rede privada).
    O acesso remoto
    facilita aos membros da equipe trabalharem de praticamente quaisquer localidades. Esta flexibilidade
    faz com que a pessoa possa participar, por exemplo, de áudio conferências e ter acesso ao e-mail mesmo estando 
    em casa (útil para se lidar com diferença de fusos horários para reuniões).

\item{Repositório de arquivos:} consiste em um espaço de uso comum para guarda de documentos compartilhados.
	O acesso ao repositório pode ser controlado de tal forma que é possível estabelecer-se níveis de acesso
	que vão de ``nenhum'', ``apenas leitura'' e ``leitura e escrita''.
	Além disso, dependendo da ferramenta utilizada, é possivel fazer
	controle de versões e edição simultânea por vários usuários.


\item{Vídeo conferência:} as formas mais comuns de vídeo conferência são por câmera acoplada ao micro-computador (as
	\textit{Webcams}) e equipamentos dedicados. As \textit{Webcams} estão popularizadas devido ao seu baixo custo e tem
	uso individual. Equipamentos dedicados de vídeo conferência são dispositivos com custo mais elevado que possuem uma 
	câmera com um sistema ótico capaz de captar a imagem de um ambiente maior (uma sala toda, ou parte dela) e 
	utiliza uma televisão ou monitor dedicado para a apresentação da imagem. Permite o uso por um grupo de pessoas.

\item{Blogs, wikis, fórums de discussão:} estas são ferramentas muito comuns na Internet e enquadram-se na categoria 
    geral de ferramentas de colaboração (e no contexto desse artigo têm uso diferente do Repositório de Arquivos
    descrito anteriormente). Todas elas disponibilizam algum tipo de interface de acesso (a maioria o faz através de uma página Web)
	e a pessoa que acessar a ferramenta deixa sua contribuição/opinião por escrito. A inserção de imagens e vídeo também
	pode ser permitida. 

\end{description}


\section{\fontsize{12}{12}\textbf{Ferramentas de Gestão de Desempenho}}

Por ferramenta de gestão de desempenho entende-se um processo formal fornecido pela empresa que envolva 
remuneração por metas atingidas (\textit{pay for performance}). 
O cumprimento das metas implicará um prêmio, assim como 
o não cumprimento implicará  alguma limitação neste mesmo benefício. 

Esse tipo de ferramenta é 
de grande valia para que o gestor de equipes remotas possa lidar com a dificuldade no acompanhamento do trabalho
executado pelos membros da equipe. A remuneração por atingimento de metas coloca o interesse pelo resultado também
nas mãos de cada membro da equipe, aliviando um pouco mais o gestor que pode reforçar o papel de autonomia que cada membro
da equipe pode exercer na tomada das decisões relativas ao dia a dia de cada um deles. 

Um exemplo de como utilizar um sistema de remuneração por atingimento de metas seria (uma extensa análise de
diversos tipos de remuneração por atingimento de metas é feita por \cite{sirota}):

\begin{description}

\item{Contrato de Desempenho:} é um documento formal escrito e acordado entre o gestor e o membro da equipe no início do ano.
                                                      O Contrato de Desempenho possui objetivos que devem ser cumpridos
                                                      pelo membro da equipe ao longo do ano. É recomendável que estes objetivos estejam escritos de
                                                      forma clara, que possuam data de entrega, que sejam factíveis e que esteja acordado entre 
                                                      as partes.

\item{Revisão Semestral:} ocorre no meio do ano e é utilizada para que o gestor e o membro da equipe revisem
                                         os objetivos para verificar quais foram atingidos e façam ajustes para o segundo período do ano.
                                         Algumas empresas fazem uma antecipação da remuneração por atingimento de metas
                                         a partir da avaliação na revisão semestral. 

\item{Avaliação de Final de Ano:} é a última etapa do ciclo de avaliação 
							       e é utilizada para que o gestor e o membro da equipe revisem o
                                                    atingimento dos objetivos acordados. O pagamento da remuneração por atingimento
                                                    de metas completa-se com esta avaliação e o membro da equipe recebe o que ainda lhe
                                                    cabe da remuneração por atingimento de metas.

\end{description}

O ciclo do contrato de desempenho se repete a cada ano.


\section{\fontsize{12}{12}\textbf{Juntando as Partes}}

Trabalhar na coordenação do grupo e na gestão individual das pessoas não depende apenas das ferramentas, mas sim em como são utilizadas. Os passos abaixo descrevem as práticas de desenvolvimento de equipes utilizadas com a aplicação das correspondentes ferramentas (\cite{winter} possui outras técnicas que também poderiam ser usadas):

\begin{enumerate}

\item Crie uma \underline{lista de distribuição} para ser utilizada em todas as comunicações com a equipe. A praticidade da lista de 
	distribuição está em não ser necessário fornecer os endereços de e-mail de cada pessoa do grupo cada vez 
	que se queira comunicar com o grupo todo, correndo o risco de se
	esquecer de alguém ou de deixar alguém de fora da comunicação porque a pessoa é
	nova no grupo e nem todos ainda se lembram de incluí-la. Esta ``exclusão por desconhecimento''
	é muito comum quando o grupo recebe e-mails de pessoas de outros departamentos
	que não necessariamente estão comunicadas do novo membro do grupo. Divulgue a lista
	e mantenha-a sempre atualizada. 

\item Avise o grupo por e-mail (use a \underline{lista de distribuição})
	que você vai fazer uma reunião por \underline{áudio conferência}. Envie um convite de \underline{calendário compartilhado} informando
	 a data, horário de início e término e os dados da áudio conferência (número de telefone, senha e dados para
	acesso à 	\underline{apresentação via Internet}). Faça desta conferência a ``reunião
    semanal da equipe'' (usada para o gestor passar ao grupo os comunicados que dizem respeito
    a todos e onde o grupo pode discutir assuntos que sejam de interesse comum).

\item Na reunião, faça uma rodada rápida entre os participantes 
		coletando os assuntos que cada um gostaria de discutir no dia. Os itens devem ser
		de interesse comum; itens particulares de cada membro da equipe serão discutidos
		em reuniões individuais que você irá marcar. Anote os itens num editor
		de texto que está sendo compartilhado com todos por uma \underline{apresentação via Internet}.
		Desta forma, cada participante pode acompanhar os assuntos em pauta e sentir-se
		mais confortável de que os assuntos de seu interesse estão na pauta de discussão (e em que momento serão discutidos).
		Ao final dessa rodada (que não deve levar mais que 5 minutos), inicie dando os seus recados,
		depois vá na sequência dando a palavra a cada um e discutindo os itens da pauta. 
		Termine a reunião no horário combinado.

\item Marque reuniões um a um: para cada membro da equipe, marque uma reunião quinzenal individual
		(\underline{áudio conferência} ou telefonema comum).
		Nessa reunião discuta assuntos relativos ao membro da equipe em questão. Como está o
		andamento dos projetos, se há alguma questão (\textit{issue}) de projeto na qual você deva intervir, questões
		de Recursos Humanos do membro da equipe, etc. Essas
		reuniões um a um são a oportunidade do gestor e o membro da equipe conhecerem-se melhor,
		e onde o gestor poderá acompanhar um pouco mais em detalhes se cada
		projeto sob responsabilidade daquele membro da equipe está indo bem e se os 
		resultados serão entregues nos prazos. 
		Utilize também as reuniões um a um para estabelecer os objetivos do \underline{Contrato de Desempenho}, 
		fazer a \underline{Revisão Semestral} e a \underline{Avaliação de Final de Ano}.


\item Crie um \underline{repositório de arquivos} para armazenas os documentos de uso comum da equipe. Isso evita o armazenamento 
	individual destes arquivos e o inevitável conflito de versões que decorre disso. 
	Comece a guardar nesse repositório todos os documentos importantes
	da equipe. Crie e divulgue o hábito de, ao invés de enviar às pessoas o documento como anexo do e-mail, armazená-lo
	no repositório indicando onde o documento se encontra para que todos possam acessá-lo. O repositório
	passará aos poucos e naturalmente a ser a referência para toda a equipe de onde encontrar
	o que precisam. O repositório ajuda a criar uma identidade de equipe, por ser um local de uso comum do grupo.


\item Caso não conheça o membro da equipe pessoalmente, obtenha aprovação e
	marque uma viagem para que vocês possam se encontrar. O grau de empatia conseguido nesse encontro facilitará muito o relacionamento
	remoto posterior. Otimize o objetivo da viagem
	com visita a fornecedores, clientes, etc. A viagem será mais produtiva depois que você já
	tiver tido algumas reuniões um a um e já conhecer um pouco mais sobre o membro da equipe.

\end{enumerate}


\section{\fontsize{12}{12}\textbf{Considerações Finais}}

Além das ferramentas descritas anteriormente existem outras que, apesar do apelo tecnológico, mostraram-se ainda 
pouco eficazes por sua baixa aceitação no uso diário da gestão de equipes remotas. 
O propósito não é desprestigiar essas ferramentas, mas apenas relatar que seu uso é, na prática,
menos adotado que o das demais.

\begin{description}

\item{Vídeo conferência:} apesar de muito comentada como substituto de viagens, a vídeo conferência não se mostrou uma
    ferramenta prática na interação dos membros da equipe. Alguns fatores que reduziram a disposição
    para o uso desta ferramenta foram:
    \begin{itemize}

    \item Nem todos os equipamentos micro-computadores (PCs) fornecidos pela empresa possuem \textit{Webcam}. Desta forma, alguns membros da equipe ficariam de fora da interação por vídeo, tornando
       a experiência menos interessante.

    \item As pessoas não se sentem confortáveis em ter sua imagem em vídeo compartilhada, especialmente
       porque o vídeo tem um efeito policiador sobre a pessoa. O interlocutor pode ver a todo momento 
       se a pessoa está ``prestando atenção'', o que nem sempre ocorre durante áudio conferências entre muitos participantes.

    \item São poucos os softwares utilizados na empresa que permitem a transmissão de vídeo. Quando muito, as ferramentas de IM
       oferecem esta funcionalidade, mas a mesma fica limitada a dois participantes apenas. Outras ferramentas também têm 
	limites de participantes simultâneos, e os dispositivos dedicados a vídeo conferência, se disponíveis, ficam alocados em salas 	
	específicas, dificultando o acesso casual sem pré-agendamento.
    \end{itemize}

\item{Blogs, wikis, fóruns de discussão:} essas ferramentas pressupõem que a pessoa vai pro-ativamente acessá-las e contribuir 
	com sua opinião, por escrito. Isso exige muita disciplina (para acessar regularmente a ferramenta) e, na prática, o e-mail acaba
	sendo o método mais utilizado para discussões.

\end{description}


Uma ressalva a ser feita a respeito de ferramentas de tecnologia é que sua utilização
depende em parte da cultura da empresa. Quem vem de muitas reuniões 
presenciais pode relutar em adotar reuniões por áudio conferência em grupo.  Trabalhar de casa por acesso remoto pode parecer ``um absurdo'' para muitos e até
mesmo criar problemas com o Departamento de Recursos Humanos ou Sindicatos. 
Mesmo assim, equipes remotas não podem prescindir do uso de ferramentas de tecnologia às custas de 
perda significativa de produtividade.


\section{\fontsize{12}{12}\textbf{Conclusões}}

A experiência na gestão de uma equipe remota (ou equipe virtual) mostrou que as ferramentas de tecnologia de
e-mail, lista de distribuição, áudio conferência, calendário compartilhado, \textit{chat}, apresentações via Internet, acesso remoto
a sistemas, e repositório de arquivos, mostraram-se de uso prático e de fácil aceitação pelo grupo.
Por outro lado, vídeo conferência, blogs, wikis e fóruns de discussão mostraram-se de baixa aceitação e menos
eficazes para uso seja por questões de cultura corporativa ou por custo/disponibilidade.
As ferramentas de Gestão de Desempenho são um complemento fundamental na administração da equipe porque
permitem ao gestor dividir com os membros do grupo o interesse pelos resultados.


\section{\fontsize{12}{12}\textbf{Áreas para Investigação}}

É provável que uma equipe remota também apresente diferenças culturais entre as pessoas, pois o fato dos membros
da equipe estarem geograficamente dispersos aumenta a chance de pertencerem a culturas diferentes.
Essas diferenças podem se manifestar até mesmo quando a dispersão geográfica ocorre dentro do mesmo país (em
estados diferentes) ou dentro do mesmo estado (por exemplo, capital versus interior).
Equipes locais também podem apresentar diferenças culturais dependendo da origem de cada membro da equipe.
Diferenças culturais adicionam um grau a mais de dificuldade na gestão de equipes remotas, no entanto
não foi objetivo deste trabalho abordar este tipo de questão.

%\theendnotes

\begin{thebibliography}{1}

  \bibitem[BRAKE 2010]{brake} BRAKE, Terence. {\em Onde está minha equipe?} 2010 : Elsevier Editora Ltda.

  \bibitem[DAVEY-WINTER 2010]{winter} DAVEY-WINTER, Karen. {\em Team Building and Development in a Matrix Environment} 2010 : PMI Virtual Library.

  \bibitem[PMBOK\textregistered 2008]{pmbok} PROJECT MANAGEMENT INSTITUTE (PMI). {\em Um Guia do Conjunto de Conhecimentos em Gerenciamento
   de Projetos (Guia PMBOK\textregistered) - Quarta Edição} 2008.

  \bibitem[SEN 2010]{sen} SEN, Kunal. {\em How to Make Virtual Teams Work Successfully} 2010 : PMI Virtual Library.

  \bibitem[SIROTA 2010]{sirota} SIROTA, David ; MISCHKIND, Louis A. ; MELTZER, Michael Irwin. {\em The Enthusiastic Employee: How 
  Companies Profit by Giving Workers What They Want} 2005 : Pearson Prentice Hall

  \end{thebibliography}


\end{document}
